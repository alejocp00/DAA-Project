\documentclass[10pt]{article}
\usepackage[utf8]{inputenc}
\usepackage[spanish]{babel}
\usepackage{listings}
\usepackage{xcolor} % Para definir colores

\lstset{
    language=Python,
    basicstyle=\ttfamily\small,
    keywordstyle=\color{blue},
    stringstyle=\color{red},
    commentstyle=\color{green},
    showstringspaces=false,
    numbers=left,
    numberstyle=\tiny\color{gray},
    breaklines=true,
    frame=single,
    captionpos=b
}


\begin{document}


\section{El Problema}

Alejandro y Semastián quieren hacer un viaje por carretera de La Habana a Guantánamo. \\
\textbf{Objetivo:} Fiesta \\
\textbf{Obstáculo:} Precio de la gasolina. Incluyendo el punto de salida (La Habana) y de destino (Guantánamo), hay un total de $n$ puntos a los que es posible visitar, unidos por $m$ carreteras cuyos costos de gasolina se conocen.\\
Los compañeros comienzan entonces a planificar su viaje. \\
\\
Luego de pensar por unas horas, Alejandro va entusiasmado hacia Semastián y le entrega una hoja. En esta hoja se encontraban $q$ tuplas de la forma $(u,v,l)$ y le explica que a partir de ahora considerarían como útiles sólo a los caminos entre los puntos $u$ y $v$ cuyo costo de gasolina fuera menor o igual a $l$, para $u$, $v$, $l$ de alguna de las $q$ tuplas. \\

Semastián lo miró por un momento y le dijo: \textit{Gracias}. La verdad esta información no era del todo útil para su viaje. Pero para no desperdiciar las horas de trabajo de Alejandro, se dispuso a buscar lo que definió como carreteras útiles. Una carretera útil es aquella que pertenece a algún camino útil. Ayude a Alejandro y Semastián encontrando el número total de carreteras útiles. \\

PD: Cuando le contaron del plan a Yisell, esta se preguntó extrañada por qué Alejandro y Semastián no habían simplemente buscado el camino de costo mínimo entre La Habana y Guantánamo. Hay que estudiar más discreta.

\section{Propuesta de análisis del ejercicio}

Para la solución que se quiere ofrecer, es necesario demostrar que dada \\ $e \in E(G)$, $e_u = <x,y> \Leftrightarrow \exists t = (u,v,l), t \in U$ tal que:
$$c(p_m(u,x)) + c(<x,y>) + c(p_m(y,v)) \leq l$$
$$\lor$$
$$c(p_m(u,y)) + c(<x,y>) + c(p_m(x,v)) \leq l$$
donde $p_m(i,j)$ es cualquier camino de costo mínimo desde $i$ a $j$, $\forall i,j \in G$

\subsection{Demostrando $\Rightarrow$ }

Sea $e \in E(G), e_u$, se tiene entonces que $e = <x,y> \in p(u,v)_u$, donde $c(p(u,v)_u)<l$, $(u,v,l) \in t$, $t \in U$. Sin pérdida de generalidad, dividamos $p(u,v)_u$ tal que $p(u,v)_u = p(u,x) \cup <x,y> \cup p(y,v)$. Se sabe por tanto, que $c(p(u,x)) + c(<x,y>) + c(p(y,v)) \leq l$ por definición. Luego cómo existe $p(u,x)$ y $p(y,v)$, existe entonces $p_m(u,x)$ y $p_m(y,v)$ respectivamente. Por tanto, $c(p_m(u,x))+c(<x,y>)+c(p_m(y,v) \leq c(p(u,x)) + c(<x,y>) + c(p(y,v)) \leq l$, por lo que $\exists p(u,v)_u = p_m(u,x) \cup <x,y> \cup p_m(y,v)$.\\

La demostración para $c(p_m(u,y)) + c(x,y) + c(p_m(x,v)) \leq l$ es análoga.

\subsection{Demostrando $\Leftarrow$}

Se tiene que $\exists t = (u,v,l), t \in U$ tal que:
$$c(p_m(u,x)) + c(<x,y>) + c(p_m(y,v)) \leq l$$
$$\lor$$
$$c(p_m(u,y)) + c(<x,y>) + c(p_m(x,v)) \leq l$$
Sin pérdida de generalidad asumamos el caso primero. Luego, se tiene que $p(u,v) = p_m(u,x) \cup <x,y> \cup p_m(y,v) = p(u,v)_u$. Luego
$$e = <x,y> \in p(u,v)_u \Rightarrow e_u$$.

De esta forma queda demostrada la doble implicación, y por tanto la base teórica.

\section{Propuesta de código}

\begin{lstlisting}[caption={Ejemplo de código Python}]
import math

def travel(G,U):
	dijkstra = None
	# Chose wish Dijkstra algorithm are going to be used
	if G.EdgesCount * math.log(G.VerticesCount) < G.VerticesCount**2:
		dijkstra = heap_dijkstra
	else:
		dijkstra = array_dijkstra
	
	# Initialize distance
	distance = {}

  

	# Apply dijkstra for any vertex in U
	
	for u,v,l in U:
		if distance.get(u) is None:
			distance[u] = dijkstra(G,u)
		if distance.get(v) is None:
			distance[v] = dijkstra(G,v)
	
	  
	
	# Find and count the util edges in the graph
	util_edge_count = 0
	for x, y, w in G.Edges:
		for u, v, l in U:
			if (distance[u][x] + distance[v][y] + w <= l or distance[u][y] + distance[v][x] + w <= l):
			util_edge_count += 1
			break
	
	return util_edge_count
\end{lstlisting}

\section{Complejidad temporal del código}

Las líneas hasta antes del primer \textit{for}, tienen una complejidad temporal de $O(1)$. En ella se inicializan las variables \textit{dijkstra} y \textit{distance}, que representan la variante del método de dijkstra a utilizar y un diccionario en el que se guardarán las distancias de los nodos calculados. Dado que el algoritmo de Dijkstra con Heap binario tiene un costo de $O(|E|log|V|)$, y el algoritmo de Dijkstra con arrays tiene un costo de $O(|V|^2)$, se hace la distinción al principio del algoritmo para garantizar que el peor caso posible sea entonces $O(|V|^2)$. Particularizando en el problema actual, se tiene entonces que el costo de aplicar Dijkstra será $O(min(|E|log|V|,|V|^2) = O(min(m*log(n),n^2))$. Luego en el primer \textit{for} se aplica Dijkstra $2q$ veces, donde $2q \leq n$. Por tanto, la complejidad temporal del primer \textit{for} sería de $O(2q*min(m*log(n),n^2))= O(q*min(m*log(n),n^2))$

Finalmente, el último for tendría una complejidad temporal de $O(m*q)$ ya que se revisan todas las aristas del grafo con todas las tuplas de $U$.

El costo total del algoritmo es de 
$$T(n,m,q) = O(1)+O(q*min(m*log(n),n^2))+O(m*q)$$
$$T(n,m,q) = O(1+q*min(m*log(n),n^2)+q*m)$$
$$T(n,m,q) = O(q*(min(m*log(n),n^2)+m))$$
Para un grafo denso, $m \approx n^2$, $min(m*log(n), n^2) = n^2$, y $O(m +n^2) = O(n^2)$. Si el grafo no es denso entonces $min(m*log(n), n^2)$ = m*log(n), y $O(m + m*log(n)) = O(m*log(n))$, por lo que quedaría en un final:
$$T(n,m,q)=O(q*min(m*log(n),n^2)$$

\end{document}