\documentclass[a4paper,12pt]{article}
\usepackage[utf8]{inputenc}
\usepackage{amsmath,amsfonts,amssymb}
\usepackage{enumitem}
\usepackage{listings}
\usepackage{geometry}
\geometry{left=25mm,right=25mm,top=25mm,bottom=25mm}

\title{Informe: MineCraft}
\date{}

\begin{document}

\maketitle

\tableofcontents

\section{Problema}

En el juego de MineCraft una de las principales distracciones es la construccion, los mejores jugadores logran hacer monumentos imponentes que sorprenden a todos. Actualmente se esta llevando a cabo un torneo de construcción en el juego, donde la tarea es hacer un muro. Un muro es, como sabemos, una hilera de columnas de bloques de piedra, todos de la misma altura. En MineCraft, para llevar a cabo esta tarea hay 3 movimientos válidos: 

- sacar un bloque de piedra del inventario y aumentar la altura de la columna en cuestión en 1 de altura 
- destruir un bloque de piedra de una columna y disminuir la altura de la columna en cuestión en 1 de altura 
- mover un bloque de piedra de una columna a otra, aumentando la altura de la 2da columna y disminuyendo la de la 1ra en 1 de altura cada una.

Se sabe que hacer cada movimiento consume c, d y m de energía respectivamente, y que en los inventarios de los jugadores hay suficientes bloques de piedra siempre. Los jugadores comienzan a jugar con un muro a medio hacer aleatorio, o sea, se les da una cantidad n de columnas de bloques de piedra en hilera, de disímiles tamaños y el ganador del torneo será el que construya un muro de largo n utilizando la menor cantidad de energía, no se permite crear columnas nuevas ni dejar huecos de antiguas columnas en el muro por supuesto. 

Elabore una estrategia que asegure que para cualquier muro inicial a medio hacer con que comience, usted logrará hacer el muro pedido utilizando la menor cantidad de energía posible.


\section{Reformulación del problema}

Dado un arreglo \texttt{heights} de longitud $n$, donde cada elemento \texttt{heights[i]} representa la altura de la columna $i$ de un muro incompleto, se debe igualar la altura de todas las columnas utilizando los siguientes movimientos, cada uno con un costo energético asociado:

\begin{itemize}
	\item Agregar un bloque a la columna $i$ (incrementa \texttt{heights[i]} en 1) con un costo de $c$ unidades de energía.
	\item Eliminar un bloque de la columna $i$ (decrementa \texttt{heights[i]} en 1) con un costo de $d$ unidades de energía.
	\item Mover un bloque de la columna $i$ a la columna $j$ (decrementa \texttt{heights[i]} en 1 y aumenta \texttt{heights[j]} en 1) con un costo de $m$ unidades de energía.
\end{itemize}

El objetivo es igualar la altura de todas las columnas del muro minimizando el costo total de energía.

\subsection{Definir la intención}

Una solución óptima del problema tiene una altura $h_0$. Todas las filas por encima de esa altura estarán vacías y todas las filas por debajo estarán llenas. Para resolver el problema, se determinará, para cada fila, su estado en dicha solución óptima. El estado de cada fila se encuentra mediante el uso de dos lemas clave:

\begin{itemize}
	\item \textbf{Lema 1:} Una fila estará llena en la solución final si el costo de rellenar los bloques hasta esa fila es menor o igual al costo de vaciar solo esa fila.
	\item \textbf{Lema 2:} Una fila estará vacía si el costo de vaciar todos los bloques hasta esa fila es menor o igual al costo de llenarla.
\end{itemize}

Se demuestra que, en cada iteración, se puede determinar el estado de al menos una fila, lo que garantiza la convergencia hacia una solución óptima.

\subsection{Observaciones}

\begin{itemize}
	\item Pueden existir varios muros óptimos con diferentes alturas.
	\item Las filas se enumeran de abajo hacia arriba, por lo tanto, si $f_0 < f_1$ entonces $f_0$ tiene menos altura que $f_1$.
	\item El término "rellenar" se refiere a construir o mover bloques hacia el lugar, y "vaciar" hace referencia a destruir o mover bloques desde el lugar.
\end{itemize}

\subsubsection*{(*1) Consideraciones sobre el movimiento de bloques}

Mover bloques hacia arriba o hacia los lados no tiene sentido. Supongamos que queremos mover de la fila $f_0$ a la fila $f_1$, siendo $f_0 \leq f_1$.

\begin{itemize}
	\item \textbf{Caso 1:} La altura óptima $h_0 \leq f_0$. En este caso, todos los bloques por encima de $h_0$, incluidos los de la fila $f_1$, tendrán que ser destruidos o movidos hacia abajo para alcanzar la altura óptima, por lo que se gasta energía innecesariamente al mover bloques.
	\item \textbf{Caso 2:} La altura óptima $h_0 \geq f_0$. En este caso, todos los bloques por encima, incluidos los de la fila $f_0$, tendrán que ser construidos para alcanzar la altura óptima, por lo que también se gasta energía innecesariamente al mover bloques.
\end{itemize}

\section{Primera solución. Correctitud}

El algoritmo es correcto si, para cualquier configuración de alturas iniciales y costos de energía $c$, $d$ y $m$, produce un costo mínimo para nivelar las alturas de las columnas. En cada iteración, evaluando el costo de cada acción, se decide si es más barato disminuir las columnas más altas o aumentar las columnas más bajas. Como esta decisión también es la mejor en una solución final, al repetir hasta que las columnas tengan la misma altura, se puede concluir que se obtiene un costo mínimo para nivelar el muro. La terminación del algoritmo está garantizada porque, como en cada iteración se reduce la diferencia entre la altura máxima $h_{\text{max}}$ y la mínima $h_{\text{min}}$, eventualmente sucederá que $h_{\text{max}} = h_{\text{min}} = h_0$.

\subsection{Lemas para determinar el estado de una fila en una solución final}

\begin{itemize}
	\item \textbf{Lema 1:} Si el costo de rellenar todos los bloques hasta una fila es menor o igual que el costo de vaciar solo esa fila, entonces la fila estará llena en alguna solución final.
	\item \textbf{Lema 2:} Si el costo de vaciar todos los bloques hasta una fila es menor o igual que el costo de rellenar una fila, entonces la fila estará vacía en alguna solución final.
\end{itemize}

\subsection{Notaciones y definiciones}

\begin{itemize}
	\item $H \rightarrow$ conjunto de alturas de soluciones óptimas.
	\item $long \rightarrow$ cantidad de columnas del muro.
	\item $c \rightarrow$ costo de construir un bloque.
	\item $d \rightarrow$ costo de destruir un bloque.
	\item $block\_count(f) \rightarrow$ cantidad de bloques de una fila $f$.
	\item $refill\_move\_count(f) \rightarrow$ cantidad de bloques que se pueden mover hacia una fila $f$.
	\item $clean\_move\_count(f) \rightarrow$ cantidad de bloques que se pueden mover desde una fila $f$.
	\item $crf(f) \rightarrow$ costo de rellenar los bloques de una fila $f$.
	\item $crtf(f) \rightarrow$ costo de rellenar todos los bloques desde el suelo hasta una fila determinada.
	\item $cvf(f) \rightarrow$ costo de vaciar los bloques de una fila $f$.
	\item $cvtf(f) \rightarrow$ costo de vaciar todos los bloques desde el tope hasta una fila determinada.
\end{itemize}

Las fórmulas para el costo de rellenar y vaciar una fila $f$ son las siguientes:

\[
crf(f) = m \times refill\_move\_count(f) + c \times (long - block\_count(f) - refill\_move\_count(f)) - d \times refill\_move\_count(f)
\]

\[
cvf(f) = m \times clean\_move\_count(f) + d \times (block\_count(f) - clean\_move\_count(f)) - c \times clean\_move\_count(f)
\]

Si $m < c + d$, los costos $crf(f)$ y $cvf(f)$ disminuirán cuando $m$ aumente. Es decir, nos interesa utilizar movimientos siempre que sea posible. Si no, los movimientos no se realizarán, ya que empeorarían el costo.

\subsection{Formalización del Lema 1}

\[
crtf(f) \leq cvf(f) \implies \exists h \in H: h \geq f
\]

Supongamos $crtf(f) \leq cvf(f)$ y no existe $h \in H: h \geq f$ para llegar a una contradicción. Si modificamos el muro óptimo de altura $h_0$, haciendo su altura igual a $f$, entonces, al rellenarlo hasta la fila $f$, la diferencia de costo con respecto al muro inicial será la de deshacer la destrucción o movimiento de los bloques que se habían eliminado y la de construir los bloques que no estaban anteriormente. Por tanto, el costo disminuye en $cvf(f)$ y aumenta en $crtf(f)$; como $crtf(f) \leq cvf(f)$, el costo del nuevo muro es menor o igual que el del muro óptimo. Por lo tanto, existe $h = f$ que pertenece a $H$, lo cual es imposible.

\subsection{Formalización del Lema 2}

\[
cvtf(f) \leq crf(f) \implies \exists h \in H: h < f
\]

Análogo al Lema 1: Supongamos $cvtf(f) \leq crf(f)$ y no existe $h \in H: h < f$ para llegar a una contradicción. Si modificamos el muro óptimo de altura $h_0$, haciendo su altura igual a $f$, entonces, al vaciarlo hasta la fila $f-1$, la diferencia de costo con respecto al muro inicial será la de deshacer la construcción o movimiento de los bloques que se habían añadido y la de destruir los bloques que estaban anteriormente. Por tanto, el costo disminuye en $crf(f)$ y aumenta en $cvtf(f)$; como $cvtf(f) \leq crf(f)$, el costo del nuevo muro es menor o igual que el del muro óptimo. Por lo tanto, existe $h = f-1$ que pertenece a $H$, lo cual es imposible.

\subsection{Existencia de una fila que cumpla alguna premisa de los lemas}

\textit{piso} $\rightarrow$ fila cuya altura es igual a la altura de la columna más baja más uno.\\
\textit{tope} $\rightarrow$ fila cuya altura es igual a la altura de la columna más alta.

\[
piso = \min(\textit{heights}) + 1
\]
\[
tope = \max(\textit{heights})
\]

Por la naturaleza de las columnas de un muro, si una fila $f_0$ está por debajo de una fila $f_1$, la cantidad de bloques en $f_0$ es mayor o igual que la cantidad de bloques en $f_1$, porque para que un bloque esté en $f_1$ necesariamente tiene que estar en $f_0$. Sin embargo, el hecho de estar en $f_0$ no implica que esté en $f_1$. De forma análoga, si falta un bloque en $f_0$, también faltará en $f_1$, pero el hecho de que falte en $f_1$ no implica que falte en $f_0$.

\[
\textbf{Lema 3:} f_0 \leq f_1 \iff \textit{block\_count}(f_0) \geq \textit{block\_count}(f_1)
\]

Ahora demostraremos que siempre se cumple la premisa del Lema 1 para la fila \textit{piso} o la del Lema 2 para la fila \textit{tope}:

\[
crtf(piso) \leq cvf(piso) \quad \text{o} \quad cvtf(tope) \leq crf(tope)
\]

\subsection{Vía 1}
Demostraremos:
\[
\neg \left( crtf(piso) \leq cvf(piso) \right) \Rightarrow cvtf(tope) \leq crf(tope)
\]
Es decir:
\[
crtf(piso) > cvf(piso) \Rightarrow cvtf(tope) \leq crf(tope)
\]

1) $crtf(piso) > cvf(piso)$\\
Sabemos que:
\[
crtf(piso) = crf(piso)
\]
por la definición de \textit{piso}, entonces tenemos:
\[
crf(piso) > cvf(piso)
\]

Además, como:
\[
crf(tope) = c \cdot (long - \textit{block\_count}(tope))
\]
ya que no se puede considerar el uso de movimientos debido a (*1).\\
En el peor de los casos, si no fuera factible usar movimientos para reducir el costo, el costo de $crf(piso)$ sería igual:
\[
crf(piso) \leq c \cdot (long - \textit{block\_count}(piso))
\]
Por el Lema 3, sabemos que:
\[
\textit{block\_count}(piso) \geq \textit{block\_count}(tope)
\]

2) Entonces se cumple que:
\[
crf(tope) \geq crf(piso)
\]

3) De lo anterior obtenemos:
\[
crf(tope) \geq crf(piso) > cvf(piso)
\]

Sabemos que:
\[
cvf(piso) = d \cdot \textit{block\_count}(piso)
\]
ya que no se puede considerar usar movimientos debido a (*1).\\
En el peor de los casos, si no fuera factible usar movimientos para reducir el costo, el costo de $cvf(tope)$ sería:
\[
cvf(tope) \leq d \cdot \textit{block\_count}(tope)
\]
Por el Lema 3, se cumple que:
\[
\textit{block\_count}(piso) \geq \textit{block\_count}(tope)
\]

4) Entonces:
\[
cvf(piso) \geq cvf(tope)
\]

Finalmente, combinamos los resultados de (3) y (4):
\[
crf(tope) \geq crf(piso) > cvf(piso) \geq cvf(tope)
\]

Dado que:
\[
cvtf(tope) = cvf(tope)
\]
por definición de \textit{tope}, tenemos que:
\[
crf(tope) \geq crf(piso) > cvf(piso) \geq cvtf(tope)
\]

Por lo tanto:
\[
crf(tope) > cvtf(tope)
\]
y concluimos:
\[
cvtf(tope) < crf(tope)
\]
Por lo tanto, se cumple que:
\[
cvtf(tope) \leq crf(tope)
\]
\textit{lo que queríamos demostrar} (lqqd).
\subsection{Vía 2}

Demostraremos que siempre se cumple:
\[
crf(piso) \leq cvf(tope) \quad \text{o} \quad cvf(tope) \leq crf(piso) \quad (p \quad \text{o} \quad \neg p)
\]

\subsubsection{Caso 1: $crf(piso) \leq cvf(tope)$}

Sabemos que:
\[
crf(piso) = crtf(piso)
\]
por definición de \textit{piso}, por lo tanto:
\[
crtf(piso) \leq cvf(tope)
\]
Además, por (4), sabemos que:
\[
cvf(tope) \leq cvf(piso)
\]
Entonces:
\[
crtf(piso) \leq cvf(tope) \leq cvf(piso)
\]
Y concluimos que:
\[
crtf(piso) \leq cvf(piso)
\]
que es la premisa del Lema 1.

\subsubsection{Caso 2: $cvf(tope) \leq crf(piso)$}

Sabemos que:
\[
cvf(tope) = cvtf(tope)
\]
por definición de \textit{tope}, por lo tanto:
\[
cvtf(tope) \leq crf(piso)
\]
Además, por (2), sabemos que:
\[
crf(piso) \leq crf(tope)
\]
Entonces:
\[
cvtf(tope) \leq crf(piso) \leq crf(tope)
\]
Y concluimos que:
\[
cvtf(tope) \leq crf(tope)
\]
que es la premisa del Lema 2.

Por lo tanto, siempre se cumple que:
\[
crtf(piso) \leq cvf(piso) \quad \text{o} \quad cvtf(tope) \leq crf(tope) \quad \text{lqqd}.
\]

\subsection{Conclusiones}
Dado que siempre se cumple la premisa del Lema 1 para la fila \textit{piso}, o la premisa del Lema 2 para la fila \textit{tope}, entonces en cada iteración del problema se puede determinar el estado de una de estas filas en una solución óptima. Esto garantiza la convergencia hacia dicha solución.

\section{Implementación de la primera solución}

Se han implementado dos soluciones basándose en la demostración:

\subsection{Solución a}

\subsubsection{Código en Python}
\begin{verbatim}
def min_energy_to_build_wall(heights, c, d, m):

destroy_count = 0
construct_count = 0
move_count = 0
move_blocks = m < c + d
max_height = max(heights)
min_height = min(heights)

while max_height != min_height:   

ceil_count = heights.count(max_height)
floor_missing_count = heights.count(min_height)

clean_possible_moves = min(ceil_count, construct_count) if move_blocks else 0
ceil_clean_cost = m * clean_possible_moves + d * (ceil_count - clean_possible_moves) - c * clean_possible_moves

refill_possible_moves = min(floor_missing_count, destroy_count) if move_blocks else 0
floor_refill_cost = m * refill_possible_moves + c * (floor_missing_count - refill_possible_moves) - d * refill_possible_moves

if ceil_clean_cost <= floor_refill_cost:
# limpiar el techo
heights = [h-1 if h == max_height else h for h in heights]
max_height -= 1
construct_count -= clean_possible_moves
move_count += clean_possible_moves
destroy_count += ceil_count - clean_possible_moves

else: 
# rellenar el suelo
heights = [h+1 if h == min_height else h for h in heights]
min_height += 1
destroy_count -= refill_possible_moves
move_count += refill_possible_moves
construct_count +=  floor_missing_count - refill_possible_moves


cost = destroy_count * d + construct_count * c + move_count * m
print("Move count ", move_count, "Destroy count ", destroy_count, "Construct count ", construct_count)
return cost
\end{verbatim}

\subsubsection{Explicación del código}

La función \texttt{min\_energy\_to\_build\_wall} tiene como objetivo igualar las alturas de las columnas de un muro, minimizando el costo energético. A continuación, se explican las principales partes del código:

\begin{itemize}
	\item La variable \texttt{move\_blocks} indica si es posible mover bloques en lugar de destruir o construir nuevos.
	\item El bucle principal se ejecuta mientras haya una diferencia entre la altura máxima y mínima.
	\item Dentro del bucle, se calculan los costos de limpiar (bajar la altura máxima) y rellenar (aumentar la altura mínima):
	\item Dependiendo de cuál costo sea menor, se decide limpiar el techo o rellenar el suelo.
	\item Al final, se calcula y se devuelve el costo total.
\end{itemize}

\subsubsection{Complejidad Temporal}

La complejidad temporal del algoritmo depende de la cantidad de iteraciones del bucle principal y de las operaciones realizadas dentro de él:

\begin{enumerate}
	\item \textbf{Bucle principal:} Este bucle se ejecuta mientras \texttt{max\_height} no sea igual a \texttt{min\_height}. En el peor de los casos, esto puede requerir un número de iteraciones igual a la diferencia entre la altura máxima y mínima, que denotaremos como \( H \).
	\item \textbf{Operaciones dentro del bucle:}
	\begin{itemize}
		\item Contar elementos con \texttt{heights.count(max\_height)} y \texttt{heights.count(min\_height)} toma \( O(n) \) en el peor de los casos, donde \( n \) es la longitud de \texttt{heights}.
		\item Las operaciones para actualizar \texttt{heights} también son \( O(n) \).
	\end{itemize}
\end{enumerate}

Por lo tanto, la complejidad temporal total es \( O(H \cdot n) \).

\subsubsection{Complejidad Espacial}

La complejidad espacial adicional del algoritmo es \( O(1) \) porque no se utilizan estructuras de datos adicionales que escalen con la entrada, ya que la lista \texttt{heights} se modifica, pero no se crea ninguna lista nueva que dependa de la entrada. El algoritmo recibe un arreglo para almacenar las alturas de las columnas, lo que requiere \( O(n) \) de espacio. La complejidad espacial total es \( O(n) \).

\subsection{Solución b}

\subsubsection{Código en Python}
\begin{verbatim}
def min_energy_to_build_wall_optimized(heights, c, d, m):

destroy_count = 0
construct_count = 0
move_count = 0
move_blocks = m < c + d
max_height =  max(heights)
min_height = min(heights)

row_block_count = [0] * (max_height + 1)

for h in heights:
row_block_count[h] += 1

for i in range(max_height, 0, -1):
row_block_count[i-1] += row_block_count[i]

while max_height != min_height:   

ceil_count = row_block_count[max_height]
floor_missing_count = len(heights) - row_block_count[min_height + 1] 

clean_possible_moves = min(ceil_count, construct_count) if move_blocks else 0
ceil_clean_cost = m * clean_possible_moves + d * (ceil_count - clean_possible_moves) - c * clean_possible_moves
refill_possible_moves = min(floor_missing_count, destroy_count) if move_blocks else 0
floor_refill_cost = m * refill_possible_moves + c * (floor_missing_count - refill_possible_moves) - d * refill_possible_moves

if ceil_clean_cost <= floor_refill_cost:
# limpiar el techo
max_height -= 1
construct_count -= clean_possible_moves
move_count += clean_possible_moves
destroy_count += ceil_count - clean_possible_moves

else: 
# rellenar el suelo
min_height += 1
destroy_count -= refill_possible_moves
move_count += refill_possible_moves
construct_count +=  floor_missing_count - refill_possible_moves


cost = destroy_count * d + construct_count * c + move_count * m
print("Move count ", move_count, "Destroy count ", destroy_count, "Construct count ", construct_count)
return cost
\end{verbatim}

\subsubsection{Explicación del Código}

El código \texttt{min\_energy\_to\_build\_wall\_optimized} optimiza el costo energético de igualar las alturas de las columnas de un muro. Solo se diferencia de la solución a en lo siguiente:

\begin{itemize}
	\item Se crea un arreglo \texttt{row\_block\_count} que guarda la cantidad de columnas de cada altura.
	\item Se rellena \texttt{row\_block\_count} para calcular cuántas columnas tienen altura mayor o igual a cada valor, iterando desde la altura máxima hacia abajo.
	\item \texttt{ceil\_count = row\_block\_count[max\_height]}: Esta línea calcula cuántas columnas tienen la altura máxima actual. \texttt{row\_block\_count[max\_height]} representa el número de columnas cuya altura es mayor o igual a \texttt{max\_height}.
	\item \texttt{floor\_missing\_count = len(heights) - row\_block\_count[min\_height + 1]}: Esta línea calcula cuántas columnas están por debajo de la altura mínima + 1 y, por lo tanto, necesitan ser rellenadas (es decir, necesitan que se les construyan más bloques). \texttt{row\_block\_count[min\_height + 1]} nos dice cuántas columnas tienen altura mayor o igual a \texttt{min\_height + 1}. Si restamos este valor del número total de columnas (\texttt{len(heights)}), obtenemos el número de columnas cuya altura es menor o igual a \texttt{min\_height}.
\end{itemize}

\subsubsection{Complejidad Temporal}

\begin{itemize}
	\item \texttt{max\_height = max(heights)} y \texttt{min\_height = min(heights)} tardan \(O(n)\), donde \(n\) es el número de columnas en \texttt{heights}.
	\item El primer bucle que recorre las alturas tiene una complejidad \(O(n)\), ya que recorre cada columna en \texttt{heights}.
	\item El segundo bucle (que ajusta los valores de \texttt{row\_block\_count} desde la altura máxima hacia abajo) tiene una complejidad de \(O(h_{\text{max}})\).
\end{itemize}

\textbf{Bucle Principal:}
Cada iteración del bucle reduce \texttt{max\_height} o incrementa \texttt{min\_height}, con un número máximo de iteraciones de \(h_{\text{max}} - h_{\text{min}}\). Dentro de cada iteración, las operaciones toman tiempo constante \(O(1)\).

Por lo tanto, la complejidad del bucle principal es \(O(h_{\text{max}} - h_{\text{min}})\).

\textbf{Complejidad total:} La complejidad total es \(O(n + h_{\text{max}})\).

\subsubsection{Complejidad Espacial}

El espacio para el arreglo \texttt{row\_block\_count}:
El espacio adicional más significativo es el arreglo \texttt{row\_block\_count}, que tiene un tamaño \(O(h_{\text{max}})\), donde \(h_{\text{max}}\) es la altura máxima de las columnas.

\textbf{Complejidad espacial total:}
Es \(O(h_{\text{max}} + n)\) teniendo en cuenta el espacio de la entrada.

\section{Solución alternativa}

\subsection{Datos}

\begin{itemize}
	\item \(n\): array de enteros que cada valor de cada $a_{i}$ representa la altura en la columna \(i\)
	\item \(c\), \(d\), \(m\): representan el gasto energético de realizar dicha acción
\end{itemize}

\subsection{Objetivo}
Encontrar una estrategia que transforme el muro inicial en un muro de altura uniforme con el mínimo costo de energía.

\subsection{Análisis}

Para cumplir con nuestro objetivo de construir un muro uniforme tenemos que saber cuál sería la altura, llamaremos \(H\) a la altura final de todas las columnas.
Este \(H\) tendrá que cumplir que:

\[min(n) \leq H \leq max(n)\]

No tendría sentido ser mayor o menor que estos ya que tendría que consumir n$\times$c energía en construir una hilera más de bloques de piedra o en el otro caso consumir n$\times$d energía en destruir una hilera.

Teniendo esto en cuenta comencemos a analizar el problema.
Dado \(H\) podemos conocer que en el muro base tenemos columnas que la sobrepasan o están por debajo de ella.

Ahora queremos conocer la diferencia de altura que existen en las columnas respecto a \(H\). Obtenemos dos valores que respresentan los bloques sobrantes y los bloques faltantes respecto a \(H\).

\[S(H) = \sum_{i=0}^{n} max(0,a_i - H)\]
\[F(H) = \sum_{i=0}^{n} max(0,H - a_i)\]

En la sumatoria nos quedamos con el máximo entre 0 y la diferencia para evitar números negativos. El 0 representaría que no sobran ni faltan bloques en la columna \(i\) dependiendo de \(S(H)\) o \(F(H)\).

Ya teniendo estos valores podemos conocer cuántos bloques se pueden mover de los sobrantes a los faltantes. Sea este valor

\[x=min(S(H),F(H))\]

\begin{itemize}
	\item \(S(H)\) $>$ \(F(H)\): estos bloques restantes no se pueden mover por lo que se tendrían que destruir.
	\item \(S(H)\) $<$ \(F(H)\): estos bloques faltantes solo se pueden rellenar extrayendo del inventario.
\end{itemize}

Mover un bloque (con gasto energético \(m\)), es destruir uno y agregar otro, por lo que se debería comprobar si es más barato o más caro mover un bloque que destruirlo y agregarlo(con gasto energético d y c respectivamente).

Si fuera más caro mover un bloque que destruirlo y agregarlo, no tendría sentido mover \(x\) bloques. Por lo que se debería comprobar.

Si \(m - c - d \) $<$ \(0\): es más barato mover bloques que destruir y agregrar por lo que deberíamos maximizar a \(x\)

Si \(m - c - d \) $\geq$ \(0\): es más barato destruir y agregar que mover bloques por lo que para minimizar el gasto \(x\) debe ser igual a \(0\).

\subsection{Función de costo}

Tomando en cuenta todo lo anterior podemos crear una función de costo en base a \(H\):

\[C(H) =  x \times m + (S(H) -  x) \times d + (F(H) - x) \times c\].

Donde podemos ver que \(x\), al representar la cantidad de bloques a mover, es multiplicado por \(m\), que es el valor energético de mover un bloque de un lugar a otro. En los otros sumandos vemos que se le resta \(x\) a \(S(H)\) y a \(F(H)\), así se conocen los bloques que faltarían por destruir y agregar respectivamente. No es posible obtener un negativo de estas sustracciones ya que x respresenta el mínimo entre \(S(H)\) y \(F(H)\), obteniendo 0 como mínimo valor.

Hallamos este costo por cada valor posible de \(H\) y nos quedamos con la altura con el menor gasto energético.

\subsection{Demostración de la optimalidad}
La estrategia es óptima porque:
\begin{itemize}
	\item Utiliza las operaciones más económicas primero: mover bloques (si m < c + d). tendrían que destruir.
	\item Minimiza el costo total para cada posible altura H.
	\item Considera todas las alturas posibles y elige la que da el menor costo total.
\end{itemize}

Al seguir este método, aseguramos que:

\begin{itemize}
	\item No hay combinación de operaciones que resulte en un menor consumo de energía.
	\item La altura H seleccionada es la que permite aprovechar al máximo los movimientos de bloques y minimizar las operaciones más costosas.
\end{itemize}

\subsection{Conclusiones}

Esta estrategia garantiza que, para cualquier muro inicial, construiremos el muro uniforme utilizando la menor cantidad de energía posible.

\subsection{Código en Python}
\begin{verbatim}
def min_energy_to_build_wall_heights_optimizado(heights, c, d, m):

best_cost = 0
best_combination = []
min_height = min(heights)
max_height = max(heights)

block_count = [0] * (max_height + 1)

for h in heights:
block_count[h] += 1

for i in range(max_height, 0, -1):
block_count[i - 1] += block_count[i]

block_count[0] = 0
for i in range(1, max_height):
block_count[i + 1] += block_count[i]

for h in range(min_height, max_height + 1):
const_count = 0
dest_count = 0
move_count = 0

dest_count = block_count[max_height] - block_count[h]
const_count = len(heights) * h - block_count[h]

if m < c + d:
move_count = min(dest_count, const_count)
dest_count -= move_count
const_count -= move_count

cost = dest_count * d + const_count * c + move_count * m

if h == min_height or cost < best_cost:
best_cost = cost
best_combination = [move_count, dest_count, const_count]

print("Best combination ", best_combination)
return best_cost
\end{verbatim}

\subsection{Explicación del código}
	
Se crea un arreglo \texttt{block\_count} para almacenar la cantidad de bloques en cada altura.

Conteo acumulado de bloques: Se actualiza \texttt{block\_count} para que en cada posición \(i\), el valor represente el número de bloques de altura \(i\) o más. Esto permite calcular cuántos bloques están por encima de cualquier altura \(h\) de forma eficiente.

Cálculo del costo para cada altura: Se exploran todas las posibles alturas objetivo \(h\) entre \texttt{min\_height} y \texttt{max\_height}. Se calculan:
\begin{itemize}
	\item \texttt{dest\_count}: el número de bloques que deben ser destruidos (bloques que están por encima de la altura \(h\)).
	\item \texttt{const\_count}: el número de bloques que deben ser construidos (bloques que faltan para alcanzar la altura \(h\)).
\end{itemize}

Si mover es más barato que construir y destruir, se intenta mover tantos bloques como sea posible. Se calcula el costo total de las operaciones y se guarda el costo si es menor que el mejor costo encontrado hasta el momento.

Al final, el algoritmo devuelve el costo mínimo.

\subsection{Complejidad temporal}
\begin{itemize}
	\item Calcular \texttt{min\_height} y \texttt{max\_height}: Esto toma \(O(n)\), donde \(n\) es el número de elementos en \texttt{heights}.
	\item Inicialización del arreglo \texttt{block\_count}: Esto toma \(O(\texttt{max\_height})\), ya que se inicializa un arreglo de longitud \(\texttt{max\_height}+1\).
\end{itemize}

\textbf{Conteo de bloques:}
\begin{itemize}
	\item El bucle que itera sobre \texttt{heights} para contar los bloques tiene un coste \(O(n)\).
	\item Luego, el ajuste del arreglo \texttt{block\_count} (sumando acumuladamente de alturas más altas a más bajas) toma \(O(\texttt{max\_height})\).
\end{itemize}

\textbf{Cálculo de costos:}

Para cada altura \(h\) entre \texttt{min\_height} y \texttt{max\_height}, el algoritmo realiza cálculos en \(O(1)\), por lo que el total es \(O(\texttt{max\_height})\).

Entonces, la complejidad temporal total es:
\[
O(n + \texttt{max\_height})
\]

\subsection{Complejidad espacial}
\begin{itemize}
	\item Arreglo \texttt{heights}: Tiene un tamaño de \(O(n)\), ya que almacena las alturas de los bloques.
	\item Arreglo \texttt{block\_count}: Tiene un tamaño de \(O(\texttt{max\_height})\).
	\item Variables auxiliares: \texttt{best\_cost}, \texttt{best\_combination}, y las variables dentro de los bucles ocupan espacio constante, es decir, \(O(1)\).
\end{itemize}

Entonces, la complejidad espacial total es:
\[
O(n + \texttt{max\_height})
\]


\section{Solución por Backtracking}


\begin{verbatim}
def min_energy_to_build_wall_backtrack(heights, c, d, m):
mask = [0] * len(heights)
return min_energy_to_build_wall_backtrackR(heights, c, d, m, 0, mask, float('inf'))

def min_energy_to_build_wall_backtrackR(heights, c, d, m, cost, mask, min_energy):

if cost >= min_energy:
return float('inf') 

if len(set(heights)) == 1:
min_energy = min(min_energy, cost)
return cost

max_height = max(heights)
min_height = min(heights)
construct_cost = float('inf')
destruct_cost = float('inf')
move_cost = float('inf')

# aumentar en 1 la altura de una de las columnas más bajas 
min_height_index = heights.index(min_height)
if mask[min_height_index] != -1:
last_mask = mask[min_height_index]
mask[min_height_index] = 1
heights[min_height_index] += 1
construct_cost = min_energy_to_build_wall_backtrackR(heights, c, d, m, cost + c, mask, min_energy)
mask[min_height_index] = last_mask
heights[min_height_index] -= 1

# disminuir en 1 la altura de una de las columnas más altas
max_height_index = heights.index(max_height)
if mask[max_height_index] != 1:
last_mask = mask[max_height_index]
mask[max_height_index] = -1
heights[max_height_index] -= 1
destruct_cost = min_energy_to_build_wall_backtrackR(heights, c, d, m, cost + d, mask, min_energy)
mask[max_height_index] = last_mask
heights[max_height_index] += 1

# mover un bloque de la columna más alta a la más baja
if mask[max_height_index] != 1 and mask[min_height_index] != -1:
last_mask_max = mask[max_height_index]
last_mask_min = mask[min_height_index]
mask[max_height_index] = -1
mask[min_height_index] = 1
heights[max_height_index] -= 1
heights[min_height_index] += 1
move_cost = min_energy_to_build_wall_backtrackR(heights, c, d, m, cost + m, mask, min_energy)
mask[max_height_index] = last_mask_max
mask[min_height_index] = last_mask_min
heights[max_height_index] += 1
heights[min_height_index] -= 1

return min(construct_cost, destruct_cost, move_cost)

\end{verbatim}

\subsection{Estructura General del Algoritmo}

\subsubsection{Base de la Recursión}
\begin{itemize}
	\item Si todos los elementos de \texttt{heights} son iguales (es decir, el muro está nivelado), el costo acumulado es la energía mínima actual.
	\item Si el costo actual supera la energía mínima registrada, el algoritmo corta esa rama de búsqueda, pues no puede producir una solución mejor.
\end{itemize}

\subsubsection{Decisiones Recursivas}
El algoritmo busca todas las posibles acciones: construir, destruir, o mover bloques. Evalúa el costo de cada acción en función del costo acumulado y decide en qué columna realizar los cambios:
\begin{itemize}
	\item Se selecciona la columna de menor altura para agregar un bloque.
	\item Se selecciona la columna de mayor altura para eliminar un bloque.
	\item Se selecciona la columna más alta para mover un bloque a la columna más baja.
\end{itemize}

\subsubsection{Máscara (mask)}
El arreglo \texttt{mask} ayuda a evitar movimientos innecesarios repetidos, limitando las acciones que se pueden tomar en columnas específicas para optimizar el proceso de backtracking.

\subsubsection{Detalles Clave}
\begin{itemize}
	\item \textbf{Incrementar altura:} Se incrementa la columna de menor altura, se ajusta el costo y se realiza la llamada recursiva.
	\item \textbf{Decrementar altura:} Se decrementa la columna de mayor altura, se ajusta el costo y se realiza la llamada recursiva.
	\item \textbf{Mover bloques:} Se transfiere un bloque de la columna más alta a la más baja y se ajusta el costo.
\end{itemize}

Finalmente, el algoritmo devuelve el costo mínimo de energía entre las tres operaciones posibles.

\subsection{Correctitud}
\subsubsection{Reducción a Subproblemas}
En cada paso recursivo, el algoritmo explora todas las posibles maneras de igualar las alturas: construir, destruir o mover bloques. Dado que explora todas las combinaciones posibles de movimientos mediante backtracking, asegura que, en algún punto, alcanzará una configuración donde todas las columnas sean iguales y retorna el costo acumulado.

\subsection{Optimalidad}

El algoritmo almacena y compara el costo mínimo de energía en cada iteración, asegurando que la solución devuelta sea óptima. Corta las ramas donde el costo supera el mínimo conocido, lo que permite minimizar el tiempo de búsqueda sin comprometer la correctitud.

\subsection{Complejidad Temporal}
El algoritmo de backtracking explora todas las combinaciones posibles de incrementos, decrementos y movimientos de bloques. Esto implica un crecimiento exponencial en el número de posibles configuraciones a explorar.

\subsubsection{Número de Posibilidades por Columna}
Para cada columna, hay tres posibles acciones: construir, destruir o mover bloques. En el peor de los casos, el algoritmo explora todas estas posibilidades, y dado que hay \( n \) columnas y el número de bloques puede cambiar en cada paso, el espacio de búsqueda es aproximadamente del orden de \( O(3^n) \).

\subsubsection{Complejidad Total}
La complejidad temporal aproximada es \( O(3^n) \), donde \( n \) es el número de columnas. Esto se debe a que en cada paso recursivo se tienen hasta tres decisiones por columna (construir, destruir o mover).

\subsection{Complejidad Espacial}
El algoritmo utiliza:
\begin{itemize}
	\item Un arreglo \texttt{mask} de tamaño \( n \) para almacenar el estado de las columnas.
	\item La pila de llamadas recursivas tiene una profundidad máxima de \( O(n) \), lo que representa el máximo número de decisiones consecutivas antes de llegar a una solución válida o descartar esa rama de búsqueda.
\end{itemize}
Por lo tanto, la complejidad espacial es \( O(n) \), donde \( n \) es el número de columnas.

\subsection{Conclusiones}
El algoritmo propuesto utiliza backtracking para encontrar la forma óptima de igualar las alturas de un muro, minimizando el costo de energía. Aunque es correcto y garantiza una solución óptima, su complejidad temporal crece exponencialmente con el número de columnas, lo que puede hacerlo impráctico para grandes valores de \( n \). Sin embargo, la técnica es efectiva para problemas pequeños y asegura la minimización del costo energético a través de la exploración exhaustiva del espacio de soluciones.

\subsubsection{Observaciones}
\begin{enumerate}
	\item Si se mueve un bloque desde una columna \( i \) con altura menor a \( h_{\text{max}} \) hacia abajo, se estaría gastando energía innecesariamente. Digamos que \( h_0 \geq h_i \), entonces en esa columna se volverá a poner un bloque. Si \( h_0 < h_i \), por las características del algoritmo cuando todas las filas con columnas más altas se hayan vaciado, \( i \) será la columna más alta y se moverá el mismo bloque hacia abajo. Da igual si no se pone en la misma posición, el gasto de energía es el mismo.
	\item Si se aumenta en 1 la altura de una columna \( i \) que no es la más baja y \( h_0 \leq h_i \), el bloque que se construyó se va a tener que destruir, incurriendo en un gasto de energía extra. Si \( h_0 > h_i \), por las características del algoritmo cuando todas las filas con columnas más bajas se hayan llenado, \( i \) será la columna más baja y se agregará un bloque en ella, aunque no sea en el mismo momento, el gasto de energía es el mismo.
	\item Si se disminuye en 1 la altura de una columna \( i \) que no es la más alta, análogo a 1 se estaría gastando energía innecesariamente, porque si \( h_0 \geq h_i \), entonces en esa columna se volverá a poner un bloque. Si \( h_0 < h_i \), por las características del algoritmo cuando todas las filas con columnas más altas se hayan vaciado, \( i \) será la columna más alta y se destruirá el mismo bloque.
\end{enumerate}


\end{document}
